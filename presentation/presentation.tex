\documentclass[c]{beamer}

%\usepackage[frenchb]{babel}
%\usepackage[T1]{fontenc}
\usepackage[utf8]{inputenc}
%\usepackage{graphicx}

\title[Projet d'Intelliigence Artificielle]{Implémentatation des Intelligences Artificielles dans le jeu Pacman }
\author{Fabien NOLLET \and Pierre P\'EZOT \and Axel RESZETKO}
\date{19 février 2014}
\institute{TELECOM Nancy}
\titlegraphic{\includegraphics[width=2.5cm]{telecom_nancy.png}}


\usetheme{Warsaw}
%\usecolortheme{spruce}

\addtobeamertemplate{footline}{\insertframenumber/\inserttotalframenumber}

\begin{document}


\begin{frame}
  \maketitle
\end{frame}

  \begin{frame}
  \tableofcontents
\end{frame}

\section{GhostCloaker}
\begin{frame}
    \frametitle{Principe}
    \begin{block}{Fonctionnalités}
        \begin{itemize}
            \item Trouver l'intersection la plus proche de Pacman atteignable en avance,
            \item Foncer sur Pacman si il est proche,
            \item Se maintenir à distance de Pacman lorsqu'il chasse les fantômes,
            \item Rentrer à la base quand ils sont morts.
        \end{itemize}
    \end{block}
\end{frame}

\begin{frame}
	\frametitle{Algorithmes}
	\begin{block}{Algorithmes}
	\begin{itemize}
        \item Parcours en largeur avec liste d'ouverts et de fermés pour la recherche de chemin,
        \item A chaque n\oe ud est associé une distance à Pacman et aux fantômes,
        \item Sélectionner et répartir les intersections les plus proches de Pacman,
	\end{itemize}
	\end{block}
\end{frame}

\begin{frame}
    \frametitle{Points forts et points faibles}
    \begin{block}{Points forts}
        \begin{itemize}
            \item Stratégie d'encerclement,
            \item Utilisation du tunnel,
            \item Fuite efficace.
        \end{itemize}
    \end{block}
    \begin{block}{Points faibles}
        \begin{itemize}
            \item Peut avoir un comportement inattendu,
            \item Répartition des intersections non optimale.
        \end{itemize}
    \end{block}
\end{frame}

\section{GhostBetter}
\begin{frame}
    \frametitle{Principe}
    \begin{block}{Fonctionnalités}
        \begin{itemize}
            \item Se répartir autour de Pacman
            \item Foncer sur Pacman si il est proche,
            \item Rentrer à la base quand ils sont chassés
            \item Rentrer à la base quand ils sont morts.
        \end{itemize}
    \end{block}
%\end{frame}

%\begin{frame}
%	\frametitle{Algorithmes}
	\begin{block}{Algorithmes}
	\begin{itemize}
        \item A chaque n\oe ud est associé une distance de Manhattan à Pacman,
        \item Algorithme A* pour la recherche de chemin, dont l'heuristique est la distance de Manhattan au Pacman.
	\end{itemize}
	\end{block}
\end{frame}

\begin{frame}
    \frametitle{Points forts et points faibles}
    \begin{block}{Points forts}
        \begin{itemize}
            \item Recherche de chemin efficace et légère,
            \item Stratégie d'encerclement,
            \item Positionnemennt à des emplacements stratégiques,
            \item Repli en cas de danger.
        \end{itemize}
    \end{block}
    \begin{block}{Points faibles}
        \begin{itemize}
            \item Non utilisation du tunnel,
            \item Repli non optimisé.
        \end{itemize}
    \end{block}
\end{frame}

\section{PacmanWTF}
\begin{frame}
    \frametitle{Principe}
    \begin{block}{Fonctionnalités}
        \begin{itemize}
            \item Chercher les Pellet et éviter les fantômes (zone de sécurité autour),
            \item Quand ce n'est pas possible, le faire sans la zone de sécurité,
            \item Chercher une SuperPellet si il y en a une proche et si Pacman ne chasse pas,
            \item Faire demi tour si un fantôme est juste devant.
        \end{itemize}
    \end{block}
\end{frame}

\begin{frame}
	\frametitle{Algorithmes}
	\begin{block}{Algorithmes}
	\begin{itemize}
        \item Parcours en largeur avec liste d'ouverts et de fermés pour la recherche de chemin,
        \item S'arrête à la première Pellet trouvée,
        \item Les fantômes et leur zone de sécurité sont considérés comme des blocs lorsqu'ils chassent.
	\end{itemize}
	\end{block}
\end{frame}

\begin{frame}
    \frametitle{Points forts et points faibles}
    \begin{block}{Points forts}
        \begin{itemize}
            \item Évite les fantômes lors de la recherche de Pellet et SuperPellet,
            \item Se protège en allant régulièrement prendre des SuperPellet,
            \item Se protège en faisant un demi-tour d'urgence si un fantôme est trop proche.
        \end{itemize}
    \end{block}
    \begin{block}{Points faibles}
        \begin{itemize}
            \item N'évite pas les encerclements,
            \item Parcours chaotique : oublie des Pellets,
            \item Ne chasse pas les fantômes.
        \end{itemize}
    \end{block}
\end{frame}




\section{Conclusion}
\begin{frame}
  \begin{block}{Conclusion}
      \begin{itemize}
          \item Mise en application concrète des algorithmes vus en cours,
          \item Reflexion sur les stratégies de jeu les plus efficaces,
          \item Quelques difficultés à appliquer la recherche de chemin sur le jeu donné.
      \end{itemize}
  \end{block}
  \begin{figure}
      \includegraphics[width=10cm]{pacman}
  \end{figure}
\end{frame}
\end{document}
